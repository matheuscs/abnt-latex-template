\chapter{Assunto Primeiro}

\section{Seção do assunto primeiro}

Formatando código inline \texttt{if / then / else} e \texttt{switch / case}.

\subsection{Subseção da segunda seção}

Como eu vou ficar quando acabar a monografia:

\begin{figure}[H]
\centerline{\includegraphics[scale=0.25]{awesome.png}}
\caption{Awesome smile}
\end{figure}

\subsubsection{Sub Sub Seção}
     
Esse é o último nível possível das seções.

Olha um texto em \textit{itálico} e um em \textbf{negrito}.


\subsection{Lorem Ipsum Dolor}\label{sec:lorem}

Nessa subseção tem uma label, olha no código, ela chama ``lorem''. Ah, e olha as aspas aqui do lado esquerdo.

Para citação direta de até três linhas basta por entre aspas e usar o cite depois. Como diria o grande pensador,
``Se tem alguma coisa que acaba com o meu dia, é a noite.'' \cite{tiririca2009dia}. Se não for uma citação direta basta não usar as aspas e usar o cite normalmente.

Já para citação direta com três ou mais linhas use o begin citacao (olhe no código).

\begin{citacao}
``Nenhuma civilização nasceu sem ter acesso a uma forma básica de alimentação e aqui nós temos uma, como também os índios e os indígenas americanos têm a deles. Temos a mandioca e aqui nós estamos e, certamente, nós teremos uma série de outros produtos que foram essenciais para o desenvolvimento de toda a civilização humana ao longo dos séculos. Então, aqui, hoje, eu tô saudando a mandioca, uma das maiores conquistas do Brasil'' \cite{rousseff2015mandioca}.
\end{citacao}

\subsection{Usando a label}

Como dito em \ref{sec:lorem}, ipsum dolor sit amet.

 
\subsection{Escrevendo código}

Um codiguinho:

\begin{lstlisting}[caption=Classe dos animais]
public class Animal {
    private String tipo;
    public Animal(String tipo) {
        this.tipo = tipo;
    }
    public void fazerBarulho() {
        if(tipo.equals("Cachorro")) {
            System.out.println("Au! Au!")
        }
        else if(tipo.equals("Gato")) {
            System.out.println("Miau!")
        }
        else if(tipo.equals("Galinha")) {
            System.out.println("Pó pó pó");
        }
    }
}
\end{lstlisting}

E uma caixa colorida:

\begin{tcolorbox}
Au! Au!\par
Miau!\par
Pó pó pó
\end{tcolorbox} 
 
